% !TEX encoding = utf8
% !TEX root = main.tex


%%%%%%%%%%%%%%%%%%%%%%%%%%%%%%%%%%%%%%%%%%%%%%%%%%%%  Esempio con molte opzioni
\documentclass[11pt,twoside,oldstyle,autoretitolo,classica,greek]{toptesi}
%%%%%%%%%%%%%%%%%%%%%%%%%%%%%%%%%%%%%%%%%%%%%%%%%%%%

% Commentare la riga seguente se si è specificata l'opzione "pdfa"
\usepackage{hyperref}

\hypersetup{%
    pdfpagemode={UseOutlines},
    bookmarksopen,
    pdfstartview={FitH},
    colorlinks,
    linkcolor={blue},
    citecolor={red},
    urlcolor={blue}
  }
%
%%%%%%% Esempio di composizione di tesi di laurea con PDFLATEX <---------------- !
%
\usepackage[utf8]{inputenc}% per macchine Linux/Mac/UNIX/Windows; sarebbe meglio utf8
\usepackage[T1]{fontenc}\usepackage{lmodern}
\usepackage[italian,english]{babel}


%%%%%%%%%%%%------PACCHETTI AGGIUNTIVI------%%%%%%%%%%%%
\graphicspath{{Immagini/}} % Specifies the directory where pictures are stored
\usepackage[square, numbers, comma, sort&compress]{natbib} 
\usepackage{subfigure}
\usepackage{tikz}
\usepackage{pgfplots}

\usepackage{fancyhdr}

\fancyhead{} % Clears all page headers and footers
\rhead{\thepage} % Sets the right side header to show the page number
\lhead{} % Clears the left side page header

\pagestyle{fancy}


%
% Per scrivere testo fasullo in "latinorum"
\usepackage{lipsum}
%

%%%%%%% Definizioni locali
\newtheorem{osservazione}{Osservazione}% Standard LaTeX


\begin{document}
\interlinea{1.5}
\ateneo{Universit\`a degli Studi di Genova}
%
%%%%%%%%%%%%%%%%%%%%%
%

%
\FacoltaDi{Facoltà di Ingegneria}

\titolo{La pressione barometrica di~Giove}% per la laurea quinquennale e il dottorato
\sottotitolo{Feature extraction for high resolution remote sensing image classification using histograms of oriented gradients}% per la laurea quinquennale e il dottorato
%

\TesiDiLaurea{Tesi di Laurea Triennale}
%%%%%%% Corso degli studi
\corsodilaurea{Ingegneria Elettronica e Tecnologie dell'Informazione}% per la laurea


%%%%%%% Per inserire la matricola rientrato sotto il nome di ogni candidato.
%\renewcommand*\IDN{\\\quad matricola: }
%
\candidato{Margherita \textsc{Piccini}}% per tutti i percorsi
\secondocandidato{Simone \textsc{Rossi}}% per la laurea magistrale solamente
\terzocandidato{Eugenio \textsc{Zuccarelli}}


\relatore{Prof.\ Gabriele Moser}% per la laurea e/o il dottorato
\secondorelatore{Dott. Vladimir Krylov}% per la laurea magistrale


%%%%%%% Seduta dell'esame
%\sedutadilaurea{Agosto 1615}% per la laurea quinquennale; oppure:
\sedutadilaurea{\textsc{Anno~accademico} 2014-2015}% per la laurea magistrale
%\esamedidottorato{Novembre 1610}% per il dottorato
%\annoaccademico{1615-1616}% solo con l'opzione classica
%\annoaccademico{2006-2007}% idem
%\ciclodidottorato{XV}% solo per il dottorato

%%%%%%% Logo della sede
\logosede[4cm]{logo}% questo e' ovviamente facoltativo, ma e' richiesto per
% il dottorato al PoliTO; in questo caso si usa il "logopolito", il nome senza
% estensione del file che contiene in forma grafica il logo.
%
%%%%%%% Per cambiare l'offset per la rilegatura; meno offset c'e', meglio e'
%\setbindingcorrection{3mm}

\iflanguage{english}{%
	\sommario{Abstract}}


\frontespizio

%\sommario

% !TEX encoding = utf8
% !TEX root = ../main.tex

\pagestyle{empty} % No headers or footers for the following pages

\null\vfill % Add some space to move the quote down the page a bit

\textit{``Thanks to my solid academic training, today I can write hundreds of words on virtually any topic without possessing a shred of information, which is how I got a good job in journalism."}

\begin{flushright}
Dave Barry
\end{flushright}
% !TEX encoding = utf8
% !TEX root = ../main.tex

\italiano
\sommario
La classificazione di immagini telerilevate è un metodo di analisi ed identificazione che permette di associare ad ogni pixel di una determinata immagine, una etichetta corrispondente ad una specifica classe e che permette ad un calcolatore di discriminare autonomamente le varie regioni costituenti l'immagine stessa. 
Le possibili applicazioni sono innumerevoli, ma questa tesi si focalizzerà su tecniche di telerilevamento di immagini ad alta risoluzione (VHRI), finalizzate all' osservazione della Terra. 
\\
Questa tesi esplora l'applicazione di una moderna tecnica di estrazione delle \emph{feature}, chiamata istogrammi di gradiente orientati (HOG), applicata a immagini multispettrali VHR.   
L'algoritmo, già largamente utilizzato nell'ambito della \emph{human detection}, ma totalmente innovativo in questo ambito, verrà analizzato dettagliatamente in ogni sua fase, evidenziando come a differenti configurazioni di parametri corrispondano variazioni nelle \emph{performance}. 
\\
Sebbene alcune combinazioni di parametri abbiano mostrato una alta discrezionalità, gli esperimenti sono risultati largamente soddisfacenti, soprattutto in relazione alla carenza di materiale preesistente in tale ambito.
Si tratta di un problema di classificazione interessante, in quanto, le regioni coinvolte sono ben diversificate, estendendosi da aree omogenee, strutture geometriche ben definite a varie zone di suolo con tessiture regolari; la principale difficoltà riscontrata in questo processo di classificazione risiede, infatti, nella sovrabbondanza del numero di classi da distinguere.
\\
I risultati sperimentali hanno dimostrato che la classificazione, tramite l'implementazione degli HOG, presenta forti limiti per quanto riguarda le classi caratterizzate da zone omogenee. Tuttavia, si sono registrati incrementi molto significativi ed altamente soddisfacenti per quanto riguarda classi con strutture geometriche ben definite.  


%Lo scopo della tesi (?) è di analizzare vari approcci applicativi, variando i valori dei parametri principali, al fine di trovare il miglior %compromesso tra accuratezza nell'identificare le tipologie di suolo e il costo computazionale che risulta essere alquanto oneroso. 
% !TEX encoding = utf8
% !TEX root = ../main.te

\english
%\chapter*{Abstract}
%\addcontentsline{toc}{chapter}{Abstract}
\sommario
\interlinea{1.3}
The last decades have seen a significant increase in the development of automatic classification techniques for digital remote sensing images, allowing this field to become appealing for the environmental monitoring and management (Earth Observation). In particular, this thesis aims at the classification of very high resolution (VHR) images, obtained by multispectral optical sensors, for the observation of urban areas.\\
The thesis explores the application of a advanced feature extraction technique, called "histogram of oriented gradients" (HOG), applied to multispectral VHR images. The algorithm, widely used in the human detection area, but new in this context of remote sensing, has been thoroughly analyzed in each phase, highlighting the correspondance between different parameter sets and different accuracy variations. \\
Three relevant cases, regarding the Amiens (France) urban area and observed through the SPOT5 sensor, have been studied.
This is an interesting and challenging classification problem, since the imaged area is well diversified, including spatially homogeneous regions (e.g. "water"), highly defined geometrical structures (e.g. "urban areas") and several ground portions with different textures (e.g. "vegetated areas").\\
The experimental results have confirmed the opportunity of additional feature extraction, associated with spatial information of the pixel magnitude, in particular in the presence of classes that strongly overlap in terms of spectral response. In the case studies, regarding the urban areas, the classification, through the HOG implementation, has presented some limitations concerning classes identified by homogeneous areas. However, there were remarkable increases regarding the highly geometrical classes.
These results suggest the potential of HOG as a feature extraction tool for urban area mapping and the opportunity to use it together with other feature analysis techniques.


%% PARTE VECCHIA
%The classification of remote sensed images is an analysis and identification method which allows to pair each pixel of a specific image with a label which corresponds to a specific class, letting a calculator to autonomously discriminate several regions which form the image itself.
%Even though there is a significant number of possible applications, this thesis will be focused on remote sensing techniques regarding VHRI, aimed to the Earth observation. 
%This thesis explores the application of a modern feature extraction technique, known as histograms of oriented gradients (HOG), applied to VHR multispectral images. 
%The algorithm, which has already been studied for what is concerned the human detection, but is totally new for this field, will be particularly analyzed in each phase, highlighting how a difference parameter set corresponds to different performance changes. 
%Even though some parameters combinations showed several differences when applied to different sets, the overall result has been highly rewarding despite  the almost total lack of previous work in this field.    
%
%Mettere la parte in mezzo????
%
%The experimental results have illustrated that the classification through the HOG implementation has several limits ragarding homogeneous regions. However, there were sharp increases in the zones charachterized by geometrical structures. 
%\interlinea{1.3}
%The classification of remote sensing images is a method of analysis and identification which allows to pair each pixel of a specific image with a label which corresponds to a specific class, letting a calculator to autonomously discriminate several regions which form the image itself.
%Although there is a significant number of possible applications, this thesis will be focused on remote sensing techniques regarding VHRI, aimed to the Earth observation. \\
%This thesis will explore the application of a modern feature extraction method, known as histograms of oriented gradients (HOG), applied to VHR multispectral images. \\
%The algorithm, which has already been studied for what is concerned human detection, but is totally new for this field, will be deeply analyzed in each phase, highlighting how a difference parameter set corresponds to different performance changes. 
%Even though some parameters combinations showed several differences when applied to different datasets, the overall result has been highly rewarding, especially referring in the almost total lack of previous work in this field.\\
%Three relevant cases will be studied, regarding SPOT5 images acquired over Amiens (France). The imaged scenes represent a challenging classification problem, involving spacially homogenous classes (e.g. "water") as well as classes with an apparent geometrical structure (e.g. "built-up land") and textured land covers (e.g. the vegetated classes).\\
%The experimental results will be discussed and turns out that the classification through the HOG implementation has several limits ragarding homogeneous regions. However, there were sharp increases in the zones charachterized by geometrical structures.

% !TEX encoding = utf8
% !TEX root = ../main.tex

\italiano
\ringraziamenti

Desideriamo ringraziare tutti coloro che ci hanno aiutato nella stesura della tesi con suggerimenti, critiche ed osservazioni.\\
Un ringraziamento particolare va al Prof. Gabriele Moser, nostro relatore, e al Dottor Vladimir Krylov, che ci hanno supportato e guidato in questi mesi. \\
Ringraziamo, inoltre, i colleghi e gli amici che ci hanno incoraggiato e accompagnato in questi anni di studio e vorremmo, infine, ringraziare le persone a noi più care, amici e famiglia, a cui questo lavoro è dedicato.


\tablespagetrue\figurespagetrue % normalmente questa riga non serve ed e' commentata

\indici


\mainmatter

% !TEX encoding = utf8
% !TEX root = ../main.tex


\chapter{CIao Title Here} % Main chapter title

\label{ChapterX} % Change X to a consecutive number; for referencing this chapter elsewhere, use \ref{ChapterX}
\fancyhead[LE,RO]{Share\LaTeX}



\lipsum[1-20]



\end{document}
