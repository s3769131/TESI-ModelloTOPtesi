% !TEX encoding = utf8
% !TEX root = main.tex


%%%%%%%%%%%%%%%%%%%%%%%%%%%%%%%%%%%%%%%%%%%%%%%%%%%%  Esempio con molte opzioni
\documentclass[11pt,twoside,oldstyle,autoretitolo,classica,greek]{toptesi}
%%%%%%%%%%%%%%%%%%%%%%%%%%%%%%%%%%%%%%%%%%%%%%%%%%%%

% Commentare la riga seguente se si è specificata l'opzione "pdfa"
\usepackage{hyperref}

\hypersetup{%
    pdfpagemode={UseOutlines},
    bookmarksopen,
    pdfstartview={FitH},
    colorlinks,
    linkcolor={blue},
    citecolor={red},
    urlcolor={blue}
  }
%
%%%%%%% Esempio di composizione di tesi di laurea con PDFLATEX <---------------- !
%
\usepackage[utf8]{inputenc}% per macchine Linux/Mac/UNIX/Windows; sarebbe meglio utf8
\usepackage[T1]{fontenc}\usepackage{lmodern}
\usepackage[italian,english]{babel}


%%%%%%%%%%%%------PACCHETTI AGGIUNTIVI------%%%%%%%%%%%%
\graphicspath{{Immagini/}} % Specifies the directory where pictures are stored
\usepackage[square, numbers, comma, sort&compress]{natbib} 
\usepackage{subfigure}
\usepackage{tikz}
\usepackage{pgfplots}

\usepackage{fancyhdr}

\fancyhead{} % Clears all page headers and footers
\rhead{\thepage} % Sets the right side header to show the page number
\lhead{} % Clears the left side page header

\pagestyle{fancy}


%
% Per scrivere testo fasullo in "latinorum"
\usepackage{lipsum}
%

%%%%%%% Definizioni locali
\newtheorem{osservazione}{Osservazione}% Standard LaTeX


\begin{document}
\interlinea{1.5}
\ateneo{Universit\`a degli Studi di Genova}
%
%%%%%%%%%%%%%%%%%%%%%
%

%
\FacoltaDi{Facoltà di Ingegneria}

\titolo{La pressione barometrica di~Giove}% per la laurea quinquennale e il dottorato
\sottotitolo{Feature extraction for high resolution remote sensing image classification using histograms of oriented gradients}% per la laurea quinquennale e il dottorato
%

\TesiDiLaurea{Tesi di Laurea Triennale}
%%%%%%% Corso degli studi
\corsodilaurea{Ingegneria Elettronica e Tecnologie dell'Informazione}% per la laurea


%%%%%%% Per inserire la matricola rientrato sotto il nome di ogni candidato.
%\renewcommand*\IDN{\\\quad matricola: }
%
\candidato{Margherita \textsc{Piccini}}% per tutti i percorsi
\secondocandidato{Simone \textsc{Rossi}}% per la laurea magistrale solamente
\terzocandidato{Eugenio \textsc{Zuccarelli}}


\relatore{Prof.\ Gabriele Moser}% per la laurea e/o il dottorato
\secondorelatore{Dott. Vladimir Krylov}% per la laurea magistrale


%%%%%%% Seduta dell'esame
%\sedutadilaurea{Agosto 1615}% per la laurea quinquennale; oppure:
\sedutadilaurea{\textsc{Anno~accademico} 2014-2015}% per la laurea magistrale
%\esamedidottorato{Novembre 1610}% per il dottorato
%\annoaccademico{1615-1616}% solo con l'opzione classica
%\annoaccademico{2006-2007}% idem
%\ciclodidottorato{XV}% solo per il dottorato

%%%%%%% Logo della sede
\logosede[4cm]{logo}% questo e' ovviamente facoltativo, ma e' richiesto per
% il dottorato al PoliTO; in questo caso si usa il "logopolito", il nome senza
% estensione del file che contiene in forma grafica il logo.
%
%%%%%%% Per cambiare l'offset per la rilegatura; meno offset c'e', meglio e'
%\setbindingcorrection{3mm}

\iflanguage{english}{%
	\sommario{Abstract}}


\frontespizio

%\sommario

% !TEX encoding = utf8
% !TEX root = ../main.tex

\pagestyle{empty} % No headers or footers for the following pages

\null\vfill % Add some space to move the quote down the page a bit

\textit{"Anyone who has never made a mistake has never tried anything new."}

\begin{flushright}
Albert Einstein 
\end{flushright}
% !TEX encoding = utf8
% !TEX root = ../main.tex

\italiano
\sommario
Qui ci va il sommario in italiano
% !TEX encoding = utf8
% !TEX root = ../main.te

\english
\sommario
Here you should write the summary

% !TEX encoding = utf8
% !TEX root = ../main.tex

\italiano
\ringraziamenti

Desideriamo ricordare tutti coloro che ci hanno aiutato nella stesura della tesi con suggerimenti, critiche ed osservazioni: a loro va la nostra gratitudine, anche se a noi spetta la responsabilità per ogni errore contenuto in questa tesi.\\
Ringrazio anzitutto il Professor Gabriele Moser, Relatore, ed il Dottor Vladimir Krylov, Correlatore: senza il loro supporto e la loro guida sapiente questa tesi non esisterebbe. \\
Un ringraziamento particolare va ai colleghi ed agli amici che ci hanno incoraggiato o che hanno speso parte del proprio tempo per leggere e discutere con noi le bozze del lavoro. \\
Vorrei infine ringraziare le persone a noi più care, amici e famiglia, a cui questo lavoro è dedicato.


\tablespagetrue\figurespagetrue % normalmente questa riga non serve ed e' commentata

\indici


\mainmatter

% !TEX encoding = utf8
% !TEX root = ../main.tex


\chapter{CIao Title Here} % Main chapter title

\label{ChapterX} % Change X to a consecutive number; for referencing this chapter elsewhere, use \ref{ChapterX}


ciao ciao ciao ciao ciao ciao ciao ciao ciao ciao ciao ciao ciao ciao ciao ciao ciao ciao ciao ciao ciao ciao ciao ciao ciao ciao ciao ciao 
ciao ciao ciao ciao ciao ciao ciao ciao ciao ciao ciao ciao ciao ciao ciao ciao ciao ciao ciao ciao ciao ciao ciao ciao ciao ciao ciao ciao 
ciao ciao ciao ciao ciao ciao ciao ciao ciao ciao ciao ciao ciao ciao ciao ciao ciao ciao ciao ciao ciao ciao ciao ciao ciao ciao ciao ciao 
ciao ciao ciao ciao ciao ciao ciao ciao ciao ciao ciao ciao ciao ciao ciao ciao ciao ciao ciao ciao ciao ciao ciao ciao ciao ciao ciao ciao 
ciao ciao ciao ciao ciao ciao ciao ciao ciao ciao ciao ciao ciao ciao ciao ciao ciao ciao ciao ciao ciao ciao ciao ciao ciao ciao ciao ciao 
ciao ciao ciao ciao ciao ciao ciao ciao ciao ciao ciao ciao ciao ciao ciao ciao ciao ciao ciao ciao ciao ciao ciao ciao ciao ciao ciao ciao 
ciao ciao ciao ciao ciao ciao ciao ciao ciao ciao ciao ciao ciao ciao ciao ciao ciao ciao ciao ciao ciao ciao ciao ciao ciao ciao ciao ciao 
ciao ciao ciao ciao ciao ciao ciao ciao ciao ciao ciao ciao ciao ciao ciao ciao ciao ciao ciao ciao ciao ciao ciao ciao ciao ciao ciao ciao 
ciao ciao ciao ciao ciao ciao ciao ciao ciao ciao ciao ciao ciao ciao ciao ciao ciao ciao ciao ciao ciao ciao ciao ciao ciao ciao ciao ciao 
ciao ciao ciao ciao ciao ciao ciao ciao ciao ciao ciao ciao ciao ciao ciao ciao ciao ciao ciao ciao ciao ciao ciao ciao ciao ciao ciao ciao 
ciao ciao ciao ciao ciao ciao ciao ciao ciao ciao ciao ciao ciao ciao ciao ciao ciao ciao ciao ciao ciao ciao ciao ciao ciao ciao ciao ciao 
ciao ciao ciao ciao ciao ciao ciao ciao ciao ciao ciao ciao ciao ciao ciao ciao ciao ciao ciao ciao ciao ciao ciao ciao ciao ciao ciao ciao 
ciao ciao ciao ciao ciao ciao ciao ciao ciao ciao ciao ciao ciao ciao ciao ciao ciao ciao ciao ciao ciao ciao ciao ciao ciao ciao ciao ciao 
ciao ciao ciao ciao ciao ciao ciao ciao ciao ciao ciao ciao ciao ciao ciao ciao ciao ciao ciao ciao ciao ciao ciao ciao ciao ciao ciao ciao 
ciao ciao ciao ciao ciao ciao ciao ciao ciao ciao ciao ciao ciao ciao ciao ciao ciao ciao ciao ciao ciao ciao ciao ciao ciao ciao ciao ciao 
ciao ciao ciao ciao ciao ciao ciao ciao ciao ciao ciao ciao ciao ciao ciao ciao ciao ciao ciao ciao ciao ciao ciao ciao ciao ciao ciao ciao 
ciao ciao ciao ciao ciao ciao ciao ciao ciao ciao ciao ciao ciao ciao ciao ciao ciao ciao ciao ciao ciao ciao ciao ciao ciao ciao ciao ciao 
ciao ciao ciao ciao ciao ciao ciao ciao ciao ciao ciao ciao ciao ciao ciao ciao ciao ciao ciao ciao ciao ciao ciao ciao ciao ciao ciao ciao 
ciao ciao ciao ciao ciao ciao ciao ciao ciao ciao ciao ciao ciao ciao ciao ciao ciao ciao ciao ciao ciao ciao ciao ciao ciao ciao ciao ciao 
ciao ciao ciao ciao ciao ciao ciao ciao ciao ciao ciao ciao ciao ciao ciao ciao ciao ciao ciao ciao ciao ciao ciao ciao ciao ciao ciao ciao 
ciao ciao ciao ciao ciao ciao ciao ciao ciao ciao ciao ciao ciao ciao ciao ciao ciao ciao ciao ciao ciao ciao ciao ciao ciao ciao ciao ciao 
ciao ciao ciao ciao ciao ciao ciao ciao ciao ciao ciao ciao ciao ciao ciao ciao ciao ciao ciao ciao ciao ciao ciao ciao ciao ciao ciao ciao 
ciao ciao ciao ciao ciao ciao ciao ciao ciao ciao ciao ciao ciao ciao ciao ciao ciao ciao ciao ciao ciao ciao ciao ciao ciao ciao ciao ciao 
ciao ciao ciao ciao ciao ciao ciao ciao ciao ciao ciao ciao ciao ciao ciao ciao ciao ciao ciao ciao ciao ciao ciao ciao ciao ciao ciao ciao 




\end{document}
