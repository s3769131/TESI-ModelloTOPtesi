% !TEX encoding = utf8
% !TEX root = ../main.tex


\chapter*{Introduzione}
\addcontentsline{toc}{chapter}{Introduzione}
\label{Introduzione}

Gli ultimi decenni hanno visto un crescente sviluppo di tecniche di telerilevamento orientate all'osservazione della Terra (\emph{Earth Observation} - EO), che permettono, oggigiorno, di avere un ampia disponibilità di immagini digitali, acquisite tramite sensori montati a bordo di satelliti o aviotrasportati.
Un ruolo chiave in tale ambito è stato giocato dal rapido aumento di missioni spaziali, volte alla messa in orbita di satelliti dotati di sensori con diverse caratteristiche, quali sensori ottici ad alta o altissima risoluzione spaziale e sensori ottici iperspettrali, capaci di fornire informazioni cartografiche e topografiche di grande dettaglio. 
Il dato satellitare, grazie alla multitemporalità e multispettralità delle sue osservazioni, rende possibile un monitoraggio ripetitivo di vaste aree geografiche, permettendo l'analisi di porzioni di suolo a scala globale, regionale e locale.\\
Data l'alta disponibilità di tali immagini ad alta risoluzione (\emph{Very High Resolution} - VHR), si è visto un crescente interesse per le procedure automatiche di classificazione, che permettono di classificare vaste porzioni di superficie terrestre in tempi sempre più brevi, risultando di estrema utilità a molteplici applicazioni di carattere ambientale,quali sviluppo urbano, riqualificazione di aree urbane inutilizzate, mappatura agricola, salvaguardia ambientale e gestione delle risorse naturali.\\
L'incremento della risoluzione spaziale nelle immagini telerilevate VHR ha introdotto, però, nuove problematiche nella classificazione. Se per risoluzioni basse o grossolane (ad esempio, intorno a $30/50$ metri) si può sovente evitare di tenere in considerazione l'informazione contestuale, ed accuratezze accettabili sono spesso ottenibili operando con i soli canali spettrali, per alte risoluzioni ($2-5$ metri) dettagli spaziali molto più precisi risultano apprezzabili nell'immagine ed una modellazione opportuna dell'informazione spaziale diventa necessario per discriminare accuratamente classi tematiche di interesse.
L'alta correlazione fra pixel adiacenti che sussiste a risoluzione così elevata ha portato allo studio di diversi metodi per tenere in considerazione, in fase di classificazione, la distribuzione spaziale delle intensità pixel.\\
Le due filosofie che sono nate come risposta a questo problema hanno visto coinvolti, da una parte, classificatori non contestuali affiancati a metodi di estrazione di \emph{feature} contestuali e, dall'altra, l'uso di metodi di classificazione contestuali, che esplicitamente incorporano l'informazione circa il contesto spaziale di ogni singolo pixel.\\
Tra i due, l'approccio che si è deciso di seguire in questa tesi è stato il primo: l'uso di una \emph{Support Vector Machine} (SVM), come classificatore non contestuale, affiancato all'algoritmo di \emph{histogram of oriented gradient} (HOG), per estrazione di \emph{feature}.
Con questa tesi, infatti, si desidera esplorare l'uso dell'algoritmo HOG di \emph{human detection}, testando l'applicabilità di questo approccio al contesto della classificazione di immagini di aree urbane multispettrali ad alta risoluzione. L'intento delle \emph{feature} HOG è quello di descrivere il comportamento locale del gradiente di un'immagine, cercando di enfatizzare, per quanto possibile, strutture geometriche ben definite; ciò è giustificato dal fatto che, intuitivamente, un oggetto possa essere identificato grazie al proprio contorno.\\
Il classificatore SVM è stato scelto in seguito alle sue proprietà di robustezza, al numero di \emph{feature} ed alla sua accuratezza in numerose applicazioni.\\

Questo tesi è organizzata nel seguente modo: nel Capitolo \ref{cap:telerilevamento} viene introdotto il contesto del telerilevamento, con accenni alle tipologie di sensori utilizzabili, ai tipi di classificatori e allo stato dell'arte delle varie strategie per l'estrazione di informazioni spaziali a fini di classificazione; nel Capitolo \ref{cap:hog}, invece, si è presentata un'analisi dettagliata dell'algoritmo di estrazione di \emph{feature} HOG introdotto da Dalal e Triggs \citep{Art_HOGHuman} con particolare riferimento alle modifiche da noi apportate necessarie per l'uso di questo approccio al contesto del telerilevamento di aree urbane. Il Capitolo \ref{cap:svm} propone la trattazione matematica della teoria alla base del classificatore SVM.
Nel Capitolo \ref{cap:risultati} si procede con una analisi dei risultati ottenuti con l'implementazione dell'algoritmo HOG e della SVM, valutandone la potenzialità, i punti di forza e debolezza e le situazioni nelle quali è vantaggioso o no utilizzarli, basandoci sugli indici di accuratezza introdotti nel Capitolo \ref{cap:prestazioni} e su sperimentazione condotta con tre \emph{dataset} reali particolarmente complessi.





%Gli ultimi decenni hanno visto un crescente sviluppo di tecniche di telerilevamento orientate all'osservazione della Terra (\emph{Earth Observation} - EO), che permettono, oggigiorno, di avere un ampia disponibilità di immagini digitali, acquisite tramite sensori montati a bordo di satelliti o aviotrasportati.
%Un ruolo chiave in tale ambito è stato giocato dal rapido aumento di missioni spaziali, volte alla messa in orbita di satelliti dotati di sensori con eccellenti caratteristiche, quali sensori ottici ad altissima risoluzione spaziale e sensori ottici iperspettrali, capaci di fornire informazioni cartografiche e topografiche di grande dettaglio. 
%Il dato satellitare, grazie alla multitemporalità e multispettralità delle sue riprese, rende possibile un monitoraggio repentino di vaste aree geografiche, permettendo l'analisi di porzioni di suolo a scala globale, regionale e locale.\\
%Data l'alta disponibilità di tali immagini ad alta risoluzione (\emph{Very High Resolution Image} - VHRI), si è visto un crescente interesse per le procedure automatiche di classificazione, che permettono di classificare vaste porzioni di superficie terrestre in tempi sempre più brevi, risultando di estrema utilità a molteplici applicazioni di carattere ambientale, quali studi della composizione del suolo, sviluppo urbano, riqualificazione di aree urbane inutilizzate, salvaguardia ambientale e gestione delle risorse naturali.\\
%L'incremento della risoluzione spaziale nelle immagini VHR ha introdotto, però, nuove problematiche nella classificazione. Se per risoluzioni basse o grossolane (approssimativamente $30/50$ metri) si può evitare di tenere in considerazione l'informazione contestuale, dal momento che pixel vicini hanno alta probabilità di condividere la stessa etichetta di classe, per alte risoluzioni ($5/2.5$ metri) questa approssimazione non si può più fare.  MOSERMOSER
%L'alta correlazione che sussiste a risoluzione così elevata ha portato allo studio di diversi metodi per tenere in considerazione, in fase di classificazione, la distribuzione spaziale dei pixel.\\
%Le due filosofie che sono nate come risposta a questo problema hanno visto coinvolti, da una parte, classificatori non contestuali affiancati a metodi di estrazione delle \emph{feature} aggiuntive e, dall'altra, l'uso di metodi di classificazione contestuali, che esplicitamente incorporano l'informazione circa il contesto spaziale di ogni singolo pixel.\\
%Tra i due, il metodo che si è deciso di seguire è stato il primo: l'uso di una SVM, come classificatore non contestuale, affiancato all'algoritmo di HOG, per estrazione di \emph{feature}.
%Con questa tesi, infatti, analizzare e descrivere l'implementazione dell'algoritmo HOG di \emph{human detection}, esplorando l'applicabilità di questo approccio al contesto della classificazione di immagini di aree urbane multispettrali ad alta risoluzione. L'intento delle \emph{feature} HOG è quello di descrivere il comportamento locale del gradiente di un'immagine, cercando di enfatizzare, per quanto possibile, strutture geometriche ben definite; ciò è giustificato dal fatto che, intuitivamente, un oggetto può essere identificato grazie al contorno.\\
%
%Questo documento è costituito nel seguente modo: nel Capitolo \ref{cap:telerilevamento} viene introdotto il problema del telerilevamento, con accenni alle tipologie di sensori utilizzabili, ai tipi di classificatori e allo stato dell'arte delle varie strategie per l'estrazione di informazioni aggiuntive per il miglioramento della classificazione; nel Capitolo \ref{cap:hog}, invece, è presente un'analisi dettagliata dell'algoritmo di \emph{detection} introdotto da Dalal e Triggs \citep{Art_HOGHuman} con particolare riferimento alle modifiche da noi apportate necessarie per l'uso di questo approccio al contesto del telerilevamento di aree urbane. Il Capitolo \ref{cap:svm} propone la trattazione matematica della teoria alla base del classificatore SVM, successivamente utilizzato per la fase di implementazione e test. 
%Nel Capitolo \ref{cap:risultati} si procede con una analisi puntuale e precisa dei risultati ottenuti con l'implementazione dell'algoritmo HOG e della SVM, valutandone la potenzialità, i punti di forza e debolezza e le situazioni nelle quali è vantaggioso o no utilizzarlo, basandoci sugli indici di accuratezza introdotti nel Capitolo \ref{cap:prestazioni}.

%La disponibilità di immagini satellitari per l'osservazione della Terra offre l'opportunità di avere immagini spazialmente e temporalmente uniformemente distribuite, di porzioni di superficie terrestre su diversa scala.\\
%Sensori passivi aviotrasportati acquisiscono immagini multispettrali con diversa risoluzione spettrale, in cui a ogni canale è associata una immagine in scala di grigio corrispondente ad un limitato intervallo di lunghezze d'onda dello spettro elettromagnetico. 
%Negli anni è aumentata la richiesta di disponibilità di mappe tematiche che definiscano la categoria a cui differenti porzioni di suolo appartengono.\\
%La costruzione automatizzata di queste carte è possibile grazie all'uso di tecniche di classificazione supervisionata di immagini.
%In quest'ottica, classificare immagini multispettrali con un così ampio numero di classi da distinguere è una sfida.\\

%MOSER STAR\\
%Negli ultimi decenni la tecnologia del telerilevamento ha acquisito un interesse crescente dal punto di vista del monitoraggio e della gestione ambientale, grazie alla copertura geografica ripetitiva che essa fornisce sul territorio. In seguito al numero crescente di missioni spaziali dedicate alla messa in orbita di satelliti per l'Osservazione della Terra ed alla relativa disponibilità di immagini del territorio, il telerilevamento presenta infatti ampie potenzialità per applicazioni ambientali su scala globale, regionale e locale.  
%MOSER END\\
%
%Per queste ragioni, si è visto un crescente interesse per le procedure automatiche di classificazione di immagini digitali acquisite tramite sensori, montati a bordo di satelliti o aviotrasportati. L'ampia disponibilità di tali immagini consente, oggigiorno, la classificazione di vaste porzioni di superficie terrestre in tempi sempre più brevi, risultando di estrema utilità a molteplici applicazioni di carattere ambientale, quali studi della composizione del suolo, sviluppo urbano, salvaguardia ambientale e gestione delle risorse naturali. L'alta risoluzione di queste immagini permette l'analisi di porzioni di suolo a diversa scala con un  livello di dettaglio elevato; tuttavia, la loro eterogeneità spaziale e spettrale rende complessa l'applicazione di semplici algoritmi di classificazione.\\ %, rendendo necessaria l'estrazione di parametri aggiuntivi al fine di migliorare i risultati.\\
%L'incremento della risoluzione spaziale nelle immagini VHR ha introdotto, però, nuove problematiche nella classificazione. Se per risoluzioni basse o grossolane (approssimativamente $30/50$ metri) si può evitare di tenere in considerazione l'informazione contestuale dovuta al fatto che pixel vicini hanno elevata probabilità di condividere la stessa etichetta di classe, per alte risoluzioni ($5/2.5$ metri) questa approssimazione non si può più fare. L'alta correlazione che sussiste a risoluzione così elevata ha portato allo studio di diversi metodi per tenere in considerazione, in fase di classificazione, la distribuzione spaziale dei pixel.\\
%Le due filosofie che sono nate come risposta a questo problema hanno visto coinvolti, da una parte, classificatori non contestuali affiancati a metodi di estrazione delle \emph{feature} aggiuntive e, dall'altra, l'uso di metodi di classificazione contestuali, che esplicitamente incorporano l'informazione circa il contesto spaziale di ogni singolo pixel.\\
%Tra i due, il metodo che si è deciso di seguire è stato il primo: l'uso di una SVM, come classificatore non contestuale, affiancato all'algoritmo di HOG, per estrazione di \emph{feature}. Con questa tesi, infatti, si vuole esplorare il campo di estrazione dei parametri aggiuntivi tramite l'analisi e l'implementazione dell'algoritmo HOG di \emph{human detection}, testando l'applicabilità di questo approccio al contesto della classificazione di immagini di aree urbane multispettrali ad alta risoluzione.\\
%
%
%
%Questo documento è costituito nel seguente modo: nel Capitolo \ref{cap:telerilevamento} viene introdotto il problema del telerilevamento, con accenni alle tipologie di sensori utilizzabili, ai tipi di classificatori e allo stato dell'arte delle varie strategie per l'estrazione di informazioni aggiuntive per il miglioramento della classificazione; nel Capitolo \ref{cap:hog}, invece, è presente un'analisi dettagliata dell'algoritmo di \emph{detection} introdotto da Dalal e Triggs \citep{Art_HOGHuman} con particolare riferimento alle modifiche da noi apportate necessarie per l'uso di questo approccio al contesto del telerilevamento di aree urbane. Il Capitolo \ref{cap:svm} propone la trattazione matematica della teoria alla base del classificatore SVM, successivamente utilizzato per la fase di implementazione e test. 
%Nel Capitolo \ref{cap:risultati} si procede con una analisi puntuale e precisa dei risultati ottenuti con l'implementazione dell'algoritmo HOG e della SVM, valutandone la potenzialità, i punti di forza e debolezza e le situazioni nelle quali è vantaggioso o no utilizzarlo, basandoci sugli indici di accuratezza introdotti nel Capitolo \ref{cap:prestazioni}.