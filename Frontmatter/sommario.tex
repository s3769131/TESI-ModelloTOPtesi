% !TEX encoding = utf8
% !TEX root = ../main.tex

\italiano
\sommario
\interlinea{1.3}
La classificazione di immagini telerilevate è un metodo di analisi ed identificazione che permette di associare ad ogni pixel di una determinata immagine, una etichetta corrispondente ad una specifica classe e che permette ad un calcolatore di discriminare autonomamente le varie regioni costituenti l'immagine stessa. 
Le possibili applicazioni sono innumerevoli, ma la nostra attenzione sarà focalizzata su tecniche di telerilevamento di immagini ad alta risoluzione (VHRI), finalizzate all'osservazione della Terra. 
\\
Questa tesi esplora l'applicazione di una moderna tecnica di estrazione delle \emph{feature}, chiamata istogrammi di gradiente orientati (HOG), applicata a immagini multispettrali VHR.   
L'algoritmo, già largamente utilizzato nell'ambito della \emph{human detection}, ma totalmente innovativo in questo contesto, verrà analizzato dettagliatamente in ogni sua fase, evidenziando come a differenti configurazioni di parametri corrispondano variazioni nelle \emph{performance}. 
\\
Sebbene alcune combinazioni di parametri abbiano mostrato una alta discrezionalità, gli esperimenti sono risultati largamente soddisfacenti, soprattutto in relazione alla carenza di materiale preesistente in tale ambito.
Saranno analizzati tre casi rilevanti, riguardanti immagini di suolo urbano di Amiens (Francia) acquisite dal sensore SPOT5; si tratta di un problema di classificazione interessante, in quanto le regioni coinvolte sono ben diversificate, estendendosi da aree spazialmente omogenee (e.g. "acqua"), strutture geometriche ben definite (e.g. zone urbane) a varie zone di suolo con \emph{texture} differenti (e.g. aree vegetate); la principale difficoltà riscontrata in questo processo di classificazione risiede, infatti, nella sovrabbondanza del numero di classi da distinguere.
\\
I risultati sperimentali hanno dimostrato che la classificazione, tramite l'implementazione degli HOG, presenta forti limiti per quanto riguarda le classi caratterizzate da zone omogenee. Tuttavia, si sono registrati incrementi molto significativi ed altamente soddisfacenti per quanto riguarda classi con strutture geometriche ben definite.  


%Lo scopo della tesi (?) è di analizzare vari approcci applicativi, variando i valori dei parametri principali, al fine di trovare il miglior %compromesso tra accuratezza nell'identificare le tipologie di suolo e il costo computazionale che risulta essere alquanto oneroso. 