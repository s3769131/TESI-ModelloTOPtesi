% !TEX encoding = utf8
% !TEX root = ../main.tex

\italiano
\sommario
\interlinea{1.3}

Gli ultimi decenni hanno visto un crescente sviluppo di tecniche di classificazione automatica di immagini telerilevate digitali, rendendo la tecnologia del telerilevamento sempre più interessante dal punto di vista del monitoraggio e della gestione ambientale (\emph{Earth Observation} - EO).
In particolare, questa tesi si è focalizzata su tecniche di classificazione di immagini ad alta risoluzione (VHR), acquisite da sensori ottici multispettrali, finalizzate all'osservazione di aree urbane. 
\\
La tesi esplora l'applicazione di una moderna tecnica di estrazione di \emph{feature}, chiamata "istogrammi di gradiente orientati" (HOG), applicata a immagini multispettrali VHR.   
L'algoritmo, già utilizzato nell'ambito della \emph{human detection}, ma innovativo nel contesto del telerilevamento, è analizzato dettagliatamente in ogni sua fase, evidenziando come a differenti configurazioni di parametri corrispondano variazioni nelle accuratezze. 
\\
Si analizzano tre casi di studio rilevanti, riguardanti immagini di aree urbane di Amiens (Francia) acquisite dal sensore SPOT5. Si tratta di un problema di classificazione interessante e alquanto complesso, dal momento che le regioni coinvolte sono ben diversificate, includendo aree spazialmente omogenee (e.g. "acqua"), strutture geometriche ben definite (e.g. zone urbane) e varie zone di suolo con tessiture differenti (e.g. aree vegetate).
\\
I risultati sperimentali hanno confermato l'opportunità dell'estrazione di \emph{feature} aggiuntive, associate all'informazione spaziale dell'intensità dei pixel, soprattutto in presenza di classi spettralmente molto sovrapposte tra loro. Si è verificato che, nei casi di studio in ambito urbano, la classificazione, tramite l'implementazione degli HOG, presenta limitazioni per quanto riguarda le classi caratterizzate da zone omogenee. Tuttavia, si sono registrati incrementi significativi per quelle classi dove è marcata la presenza di strutture geometriche regolari.
Ciò suggerisce le potenzialità di HOG come strumenti di estrazione di \emph{feature} per mappature di aree urbane e ne suggerisce l'uso insieme ad altre tecniche di analisi di tessitura.



%% VECCHIE VERSIONI
%La classificazione di immagini telerilevate è un metodo di analisi ed identificazione che permette di associare ad ogni pixel di una determinata immagine, una etichetta corrispondente ad una specifica classe e che permette ad un calcolatore di discriminare autonomamente le varie regioni costituenti l'immagine stessa. Gli ultimi decenni hanno visto un crescente sviluppo di tecniche di classificazione automatica di immagini digitali, rendendo la tecnologia del telerilevamento sempre più interessante dal punto di vista del monitoraggio e della gestione ambientale (\emph{Earth Observation} - EO).
%In particolare, la nostra attenzione sarà focalizzata su tecniche di classificazione di immagini ad alta risoluzione (VHRI), acquisite da sensori ottici multispettrali, finalizzate all'osservazione di aree urbane. 
%\\
%Questa tesi esplora l'applicazione di una moderna tecnica di estrazione delle \emph{feature}, chiamata istogrammi di gradiente orientati (HOG), applicata a immagini multispettrali VHR.   
%L'algoritmo, già largamente utilizzato nell'ambito della \emph{human detection}, ma totalmente innovativo in questo contesto, verrà analizzato dettagliatamente in ogni sua fase, evidenziando come a differenti configurazioni di parametri corrispondano variazioni nelle \emph{performance}. 
%\\
%Saranno analizzati tre casi rilevanti, riguardanti immagini di suolo urbano di Amiens (Francia) acquisite dal sensore SPOT5; si tratta di un problema di classificazione interessante e alquanto complesso, dal momento che le regioni coinvolte sono ben diversificate, estendendosi da aree spazialmente omogenee (e.g. "acqua"), strutture geometriche ben definite (e.g. zone urbane) a varie zone di suolo con \emph{texture} differenti (e.g. aree vegetate); la principale difficoltà riscontrata in questo processo di classificazione risiede, infatti, nella sovrabbondanza del numero di classi da distinguere.
%Sebbene la presenza di effetti di sfocatura nelle immagini, di ombre in aree urbane e di coperture di suolo poco discriminabili abbia reso ulteriormente complesso il processo di classificazione, gli esperimenti sono risultati largamente soddisfacenti, soprattutto in relazione alla carenza di materiale preesistente in tale ambito.
%\\
%I risultati sperimentali hanno confermato l'opportunità dell'estrazione di \emph{feature} aggiuntive, associate all'informazione spaziale dell'intensità dei pixel, soprattutto in presenza di  classi spettralmente molto sovrapposte tra loro. Si è infine dimostrato che nei casi di studio in ambito urbano la classificazione, tramite l'implementazione degli HOG, presenta forti limiti per quanto riguarda le classi caratterizzate da zone omogenee. Tuttavia, si sono registrati incrementi molto significativi ed altamente soddisfacenti per quelle classi dove è marcata la presenza di strutture geometriche regolari.
%La classificazione di immagini telerilevate è un metodo di analisi ed identificazione che permette di associare ad ogni pixel di una determinata immagine, una etichetta corrispondente ad una specifica classe e che permette ad un calcolatore di discriminare autonomamente le varie regioni costituenti l'immagine stessa. 
%Le possibili applicazioni sono innumerevoli, ma la nostra attenzione sarà focalizzata su tecniche di telerilevamento di immagini ad alta risoluzione (VHRI), finalizzate all'osservazione della Terra. 
%\\
%Questa tesi esplora l'applicazione di una moderna tecnica di estrazione delle \emph{feature}, chiamata istogrammi di gradiente orientati (HOG), applicata a immagini multispettrali VHR.   
%L'algoritmo, già largamente utilizzato nell'ambito della \emph{human detection}, ma totalmente innovativo in questo contesto, verrà analizzato dettagliatamente in ogni sua fase, evidenziando come a differenti configurazioni di parametri corrispondano variazioni nelle \emph{performance}. 
%\\
%Sebbene alcune combinazioni di parametri abbiano mostrato una alta discrezionalità, gli esperimenti sono risultati largamente soddisfacenti, soprattutto in relazione alla carenza di materiale preesistente in tale ambito.
%Saranno analizzati tre casi rilevanti, riguardanti immagini di suolo urbano di Amiens (Francia) acquisite dal sensore SPOT5; si tratta di un problema di classificazione interessante, in quanto le regioni coinvolte sono ben diversificate, estendendosi da aree spazialmente omogenee (e.g. "acqua"), strutture geometriche ben definite (e.g. zone urbane) a varie zone di suolo con \emph{texture} differenti (e.g. aree vegetate); la principale difficoltà riscontrata in questo processo di classificazione risiede, infatti, nella sovrabbondanza del numero di classi da distinguere.
%\\
%I risultati sperimentali hanno dimostrato che la classificazione, tramite l'implementazione degli HOG, presenta forti limiti per quanto riguarda le classi caratterizzate da zone omogenee. Tuttavia, si sono registrati incrementi molto significativi ed altamente soddisfacenti per quanto riguarda classi con strutture geometriche ben definite.  


%Lo scopo della tesi (?) è di analizzare vari approcci applicativi, variando i valori dei parametri principali, al fine di trovare il miglior %compromesso tra accuratezza nell'identificare le tipologie di suolo e il costo computazionale che risulta essere alquanto oneroso. 