% !TEX encoding = utf8
% !TEX root = ../main.te

\english
\sommario

The classification of remote sensed images is an analysis and identification method which allows to pair each pixel of a specific image with a label which corresponds to a specific class, letting a calculator to autonomously discriminate several regions which form the image itself.
Even though there is a significant number of possible applications, this thesis will be focused on remote sensing techniques regarding VHRI, aimed to the Earth observation. 
This thesis explores the application of a modern feature extraction technique, known as histograms of oriented gradients (HOG), applied to VHR multispectral images. 
The algorithm, which has already been studied for what is concerned the human detection, but is totally new for this field, will be particularly analyzed in each phase, highlighting how a difference parameter set corresponds to different performance changes. 
Even though some parameters combinations showed several differences when applied to different sets, the overall result has been highly rewarding despite  the almost total lack of previous work in this field.    

Mettere la parte in mezzo????

The experimental results have illustrated that the classification through the HOG implementation has several limits ragarding homogeneous regions. However, there were sharp increases in the zones charachterized by geometrical structures. 