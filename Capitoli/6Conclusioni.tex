% !TEX encoding = utf8
% !TEX root = ../main.tex


\chapter{Conclusioni}
\label{cap:conclusioni}

In questa tesi si è affrontato il problema della classificazione di immagini telerilevate ad alta risoluzione (VHR), con particolare attenzione al contributo che l'estrazione di \emph{feature} aggiuntive può apportare all'accuratezza dei risultati ottenuti sulle mappe di classificazione.
In particolare è stata esplorata l'opportunità di usare l'approccio HOG proposto inizialmente in altre applicazioni di computer vision. Tale approccio è stato analizzato e implementato analizzato e implementato, testandone l'applicabilità al problema della mappatura del suolo in aree urbane tramite immagini multispettrali telerilevate.\\
Il metodo HOG di estrazione di \emph{feature} è stato cambiata con un classificatore non parametrico SVM e con un metodo di ottimizzazione automatica dei parametri di tale classificatore. Tale approccio è stato sperimentato con tre casi di studio, in cui l'eterogeneità spaziale e spettrale delle immagini coinvolte rende complesso il problema di classificazione. In particolare, le immagini utilizzate hanno presentato sfuocatura dovuta a risoluzione spaziale meno fine della dimensione del pixel, a solo tre bande spettrali e presenze di occlusioni (ombre) nelle aree urbane coinvolte.
Durante le sperimentazioni sono sorti diversi problemi, difficilmente individuabili a priori. \\
In primo luogo, la presenza di sfuocatura nei \emph{dataset} a $2.5$ metri dovuta a tecniche di \emph{pansharpening} non perfettamente ottimizzate hanno reso la classificazione delle immagini a più alta risoluzione più complessa di quanto preventivato.\\
I valori di accuratezza ottenuti non sono risultati elevati, tuttavia si sono dimostrati accettabili e soddisfacenti, soprattutto in relazione alla difficoltà nel discriminare classi di uso del suolo coi dati disponibili carenza di materiale preesistente in tale ambito. 
Infatti, confrontando i parametri di accuratezza ottenuti con e senza estrazione di \emph{feature}, si è potuto constatare come l'utilizzo dell'algoritmo HOG abbia spesso portato ad un incremento di accuratezza, soprattutto per quelle classi per le quali è marcata la presenza di strutture geometriche regolari. 
Quest'approccio ha, invece, mostrato limitazioni nella mappatura di aree spazialmente omogenee.\\
Per queste ragioni, nei casi di studio in ambito urbano, le \emph{feature} di tessitura HOG sono uno strumento efficace per zone caratterizzate da un elevato livello di dettaglio geometrico, quali zone urbane e strade, mentre è consigliabile affiancare ad esse altri metodi di \emph{feature extraction} per classificare con affidabilità porzioni di suolo strutturalmente omogenee.\\
Esempi di futuri sviluppi in tale ambito potrebbero ad esempio riguardare l'utilizzo di HOG insieme non solo ai canali spettrali ma anche ad altre tipologie di \emph{feature} quali i parametri morfologici, il semivariogramma oppure associare tale algoritmo ad informazione di segmentazione, in modo da integrare nel processo di classificazione informazione complementare a quella fornita dai descrittori HOG qui esposti.


%%  VECCHIE VERSIONI
%In questa tesi si è affrontato il problema della classificazione di immagini ad alta risoluzione (VHRI), con una particolare attenzione al contributo che l'estrazione di \emph{feature} aggiuntive può apportare all'accuratezza dei risultati ottenuti sulle mappe di classificazione.
%In particolare è stato analizzato e implementato l'algoritmo HOG, testandone l'applicabilità al problema della mappatura del suolo tramite immagini multispettrali telerilevate.\\
%Nella parte di sperimentazione sono stati presentati tre casi di studio, in cui l'eterogeneità spaziale e spettrale delle immagini coinvolte ha reso difficoltosa la classificazione.
%Durante le sperimentazioni sono sorti diversi problemi, difficilmente individuabili a priori. \\
%In primo luogo, la presenza di sfuocatura nei \emph{dataset} a $2.5$ metri dovuta a tecniche di \emph{pansharpening} non perfettamente ottimizzate hanno reso la classificazione delle immagini a più alta risoluzione più complessa di quanto preventivato.\\
%La presenza, invece, di sole tre bande spettrali ha pregiudicato in maniera marcata i risultati ottenuti nella classe "strada" in zona urbana per tutti e tre i \emph{dataset}. Questo è dovuto al fatto che la risposta spettrale delle strade, ancor più se in ombra (come accade nell'area urbana) è molto simile a quella dell'acqua. \'E interessante notare come, però, l'inserimento delle \emph{feature} HOG abbia giocato un ruolo chiave nella classificazione di questa classe; si sono registrati, infatti, picchi di accuratezza non eccezionali (circa $12\%$) ma che, se confrontati con quelli registrati senza estrazione di \emph{feature} ($0.5/1\%$), rappresentano un notevole passo in avanti. 
%In particolare, la presenza in tutti e tre i \emph{dataset} di effetti di sfocatura, presenza marcata di ombre in aree urbane e classi tematiche talora caratteristiche di uso del suolo e non di copertura del suolo, e quindi poco discriminabili, hanno causato problemi di classificazione. \\
%Ultimo punto critico di questi \emph{dataset} è da ricercare nell'elevato numero di classi da identificare: molti errori commessi in fase di classificazione sono nati dalla difficoltà di etichettare correttamente coperture di suolo molto simili (come aree urbane di diverse densità) o praticamente identiche (come fiumi e specchi d'acqua).\\
%
%In conclusione, quindi, i valori di accuratezza ottenuti non sono risultati significativamente elevati, tuttavia si sono dimostrati accettabili e largamente soddisfacenti, soprattutto in relazione alla carenza di materiale preesistente in tale ambito. 
%Infatti, confrontando i parametri di precisione ottenuti con e senza estrazione di \emph{feature}, si è potuto constatare come l'utilizzo dell'algoritmo HOG abbia spesso portato ad un incremento di accuratezza, soprattutto per quelle classi dove è marcata la presenza di strutture geometriche regolari. 
%Quest'approccio ha, invece, mostrato i suoi limiti nella mappatura di aree spazialmente omogenee.\\
%Per queste ragioni, nei casi di studio in ambito urbano, le \emph{feature} di tessitura HOG sono molto soddisfacenti per quelle zone caratterizzate da un elevato livello di dettaglio e geometricità, quali zone urbane e strade, mentre è consigliabile affiancare altri metodi di \emph{feature extraction} per classificare con affidabilità porzioni di suolo strutturalmente omogenee. Esempi di sviluppo in tale ambito potrebbero ad esempio riguardare l'utilizzo di HOG insieme non solo ai canali spettrali ma anche ad altre tipologie di \emph{feature} quali i parametri morfologici, i semivariogrammi oppure associare tale algoritmo ad informazione di segmentazione, in modo da compensare le carenze presentatesi in alcune classi, e incrementando cosi l'accuratezza della classificazione.

%In questa tesi si è affrontato il problema della classificazione di immagini ad alta risoluzione (VHRI), con una particolare attenzione al contributo che l'estrazione di \emph{feature} aggiuntive può apportare all'accuratezza dei risultati ottenuti sulle mappe di classificazione.
%In particolare è stato analizzato e implementato l'algoritmo HOG, testandone l'applicabilità al problema della mappatura del suolo da immagini multispettrali telerilevate.\\
%Nella parte di sperimentazione sono stati presentati tre casi di studio, in cui l'eterogeneità spaziale e spettrale delle immagini coinvolte ha reso difficoltosa la classificazione. In particolare, confrontando i parametri di precisione ottenuti con e senza estrazione di \emph{feature}, si è potuto constatare come l'utilizzo dell'algoritmo HOG abbia spesso portato un incremento di accuratezza, soprattutto per  quelle  classi dove è marcata la presenza di strutture geometriche regolari. Quest'approccio ha invece mostrato i suoi limiti nella mappatura di aree spazialmente omogenee.\\
%Per queste ragioni, nei casi di studio in ambito urbano, le \emph{feature} di tessitura HOG sono molto soddisfacenti per quelle zone caratterizzate da un elevato livello di dettaglio e geometricità, mentre è consigliabile affiancare altri metodi di \emph{feature extraction} per classificare con affidabilità porzioni di suolo strutturalmente omogenee.\\
%In generale, spazi delle \emph{feature} ad alta dimensionalità hanno reso possibile la discriminazione di classi con differenze anche molto sottili con un elevato livello di dettaglio. Tuttavia, volumi così alti hanno fatto sì che ottenere livelli di precisione adeguata sulle numerose classi in gioco  fosse una ardua sfida. Al crescere della dimensione dello spazio delle \emph{feature} aumenta, infatti, anche il numero di parametri del classificatore, rendendo il numero dei pixel di training (che è fissato dal particolare \emph{dataset}) insufficienti ad effettuare stime accettabili. Questo è emerso dai numerosi risultati ottenuti, dove si è potuto osservare che a classi aventi un numero di pixel di training limitati sono corrisposti valori di PA estremamente bassi.  Si deve tener conto che  il numero dei campioni di training necessari ad addestrare un classificatore con dati ad alta dimensionalità è molto più grande di quello per dati convenzionali. In quest'ottica la limitata disponibilità di informazione del \emph{training set}, dovuta al fatto che ottenere questi campioni è difficile e costoso,  porta a problematiche non trascurabili per quanto riguarda i parametri di accuratezza.\\
%In conclusione quindi, possibili miglioramenti dell'algoritmo proposto possono essere ottenuti tramite la riduzione del numero di \emph{feature} utilizzate (\emph{feature reduction}). Questa può  essere ottenuta sia mediante tecniche che identificano un opportuno sottoinsieme dei parametri disponibili (\emph{feature selection}), sia trasformando lo spazio $d$-dimensionale  in uno spazio a dimensione minore tramite una funzione che minimizzi la relativa perdita di informazione (\emph{feature extraction}).